\documentclass{article}

% To compile the pdf, execute the following in the terminal:
% $ pdflatex ${file_name}.tex
% $ bibtex ${file_name}
% $ sage ${file_name}.sagetex.sage
% $ pdflatex ${file_name}.tex

\usepackage{sagetex}
\setlength{\sagetexindent}{10ex}

\usepackage{hyperref}
\hypersetup{
    colorlinks,
    citecolor=blue,
    filecolor=black,
    linkcolor=blue,
    urlcolor=blue,
}

\usepackage{amsmath,amsthm,amssymb,mathtools,bm}

\usepackage[margin=1in]{geometry}


\numberwithin{equation}{section}

\newlength\tindent
\setlength{\tindent}{\parindent}
\setlength{\parindent}{0pt}
\renewcommand{\indent}{\hspace*{\tindent}}


\newcommand{\makered}[1]{{\color{red}#1}}

\newcommand{\vect}[1]{\mathbf{{#1}}}
\newcommand{\mat}[1]{\mathbf{{#1}}}

\title{Bezier Bases for Simplex and Pyramid Elements}
\author{Philip Zwanenburg}

\begin{document}
\maketitle

%%%%%%%%%%%%%%%%%%%%%%%%%%%%%%%%%%%%
%%% Begin Modifiable parameters. %%%
%%%%%%%%%%%%%%%%%%%%%%%%%%%%%%%%%%%%

\begin{sagesilent}
dim = 3

# Options: regular, standard.
type_ref = "regular"
#type_ref = "standard"
\end{sagesilent}

%%%%%%%%%%%%%%%%%%%%%%%%%%%%%%%%%%%%
%%% End Modifiable parameters.   %%%
%%%%%%%%%%%%%%%%%%%%%%%%%%%%%%%%%%%%


The procedure for the derivation of the basis functions is taken directly from Chan et al.~\cite{Chan2016_bez}.

\begin{sagesilent}
var('r,s,t')
var('a,b,c')
var('i,j,k,l')
assume(i,j,k,l,'integer')
\end{sagesilent}

\section{Simplices}

Given the coordinates of the vertices of the reference triangle,
\begin{sagesilent}
if (type_ref == "regular"):
    rst_V = matrix(dim+1,dim,[-1,-1/sqrt(3),-1/sqrt(6),
                               1,-1/sqrt(3),-1/sqrt(6),
                               0, 2/sqrt(3),-1/sqrt(6),
                               0, 0,         3/sqrt(6) ])
elif (type_ref == "standard"):
    rst_V = matrix(dim+1,dim,[ 0,0,0,
                               1,0,0,
                               0,1,0,
                               0,0,1 ])
\end{sagesilent}

\[
\bm{r}_V = \sage{rst_V},
\]

the barycentric coordinates can be found by solving the following linear system
\begin{sagesilent}
A = matrix(SR,dim+1,dim+1)
A[0,:] = matrix(1,dim+1,[1]*(dim+1))
A[1:,:] = rst_V.T
l_rhs = matrix(SR,dim+1,1,[1,r,s,t])
l_rst = A\l_rhs
\end{sagesilent}

\[
\mat{A} \vect{\lambda} = \vect{b}
\]

where
\[
\mat{A} = \sage{A},\ \vect{b} = \sage{l_rhs}.
\]

The result is
\[
\vect{\lambda} = \sage{l_rst}.
\]


\begin{sagesilent}
if (type_ref == "standard"):
    rst_to_abc = [a == r/(1-(s+t)), b == s/(1-t), c == t]
elif (type_ref == "regular"):
    rst_to_abc = [a == 6*r/(3-2*sqrt(3)*s-sqrt(6)*t), b == 1/3*(8*sqrt(3)*s/(3-sqrt(6)*t)-1), c == 1/2*(sqrt(6)*t-1)]

s_abc = solve(rst_to_abc,a,b,c)
abc = matrix(SR,dim,1,[rst_to_abc[n].rhs() for n in range(0,dim)])

s_rst = solve(rst_to_abc,r,s,t)
rst = matrix(SR,dim,1,[s_rst[0][n].rhs() for n in range(0,dim)]).apply_map(lambda x: x.factor())

l_abc = l_rst(r = rst[0][0],s = rst[1][0], t = rst[2][0]).apply_map(lambda x: x.factor())
\end{sagesilent}

Given the Duffy-type transform mapping the reference coordinates to the $d$-cube,
\[
\vect{a} = \sage{abc},
\]

these equations can be solved for the representation of the $rst$ coordinates in terms of the $abc$ coordinates, 
\begin{equation} \label{eq:abc_to_rst}
\vect{r} = \sage{rst},
\end{equation}

with the barycentric coordinates then given by
\begin{equation} \label{eq:bcoords_abc}
\vect{\lambda} = \sage{l_abc}.
\end{equation}

The $ijkl^{\text{th}}$ Bezier basis of order $p$ for the simplex is given by
\[
B^p_{ijkl} \coloneqq C^p_{ijkl} \lambda_0^i \lambda_1^j \lambda_2^k \lambda_3^l,
\]

where $i+j+k+l = p$ and
\[
C^p_{ijkl}
= \frac{p!}{i!j!k!l!}
=
\frac{(i+j)!}{i!j!}
\frac{(i+j+k)!}{(i+j)!k!}
\frac{(i+j+k+l)!}{(i+j+k)!l!}.
\]

After substitution of~\eqref{eq:bcoords_abc}, we obtain
\begin{sagesilent}
B_p = (l_abc[0][0]^i*l_abc[1][0]^j*l_abc[2][0]^k*l_abc[3][0]^l).factor()

# Note these are the same as l_abc after substituting the left-most values of b and c.
if (type_ref == "standard"):
    abc_l = 0
elif (type_ref == "regular"):
    abc_l = -1
l_a = matrix(SR,2,1,[l_abc[n][0](b=abc_l,c=abc_l) for n in range(0,2)])
\end{sagesilent}
\[
B^p_{ijkl}(a,b,c) = C^p_{ijkl} \sage{B_p}.
\]

Noting the definition of the $i$th 1D Bezier basis function of degree $p$,
\begin{equation} \label{eq:bezier_1d}
B^p_i(a) = \frac{p!}{i!(p-i)!} (\sage{l_a[0][0]})^{p-i} (\sage{l_a[1][0]})^{i},
\end{equation}

it is then possible to represent the simplex basis using a tensor-product of 1D Bezier basis functions as
\[
B^p_{i,j,k,l}(a,b,c) = B^{i+j}_j(a) B^{i+j+k}_k(b) B^{i+j+k+l}_l(c).
\]

As the gradients of the basis are taken with respect to the $rst$ coordinates, we first note that the Jacobian of $abc$
coordinates with respect to the $rst$ coordinates is given by
\begin{sagesilent}
da_dr = matrix(SR,dim,dim)
da_dr[:,0] = diff(abc,r)
da_dr[:,1] = diff(abc,s)
da_dr[:,2] = diff(abc,t)

da_dr_simp = matrix(SR,dim,dim)
for n in range(0,dim):
    for m in range(0,dim):
        if (da_dr[n,m] != 0):
            da_dr_simp[n,m] = (da_dr[n,m](r = rst[0][0],s = rst[1][0], t = rst[2][0])).rational_simplify().factor()

da_dr_con = matrix(SR,dim,dim)
da_dr_con[0,:] = l_abc[0](a=abc_l)                *da_dr_simp[0,:]
da_dr_con[1,:] = l_abc[0](a=abc_l,b=abc_l)        *da_dr_simp[1,:]
da_dr_con[2,:] = l_abc[0](a=abc_l,b=abc_l,c=abc_l)*da_dr_simp[2,:]
\end{sagesilent}

\[
\mat{\frac{d\vect{a}}{d\vect{r}}} = \sage{da_dr},
\]

and after substituting~\eqref{eq:abc_to_rst}, by
\[
\mat{\frac{d\vect{a}}{d\vect{r}}} = \sage{da_dr_simp}.
\]

It is important to note that these Jacobian terms are singular at degenerate points of the mapping and that the basis
gradients must be defined accordingly. Observing that the singular component will always affect the first term
in~\eqref{eq:bezier_1d}, we note that
\begin{equation*}
\frac{p!}{i!(p-i)!} (\sage{l_a[0][0]})^{p-i-1} (\sage{l_a[1][0]})^{i} = \frac{p}{p-i} B^{p-1}_i(a),
\end{equation*}

which is used in defining the gradients of $B^p_{i,j,k,l}(a,b,c)$ with respect to the $rst$ coordinates as
%B^p_{i,j,k,l}(a,b,c) = B^{i+j}_j(a) B^{i+j+k}_k(b) B^{i+j+k+l}_l(c).
\begin{align*}
\frac{\partial B^p_{i,j,k,l}(a,b,c)}{\partial r}
& = \left(\sage{da_dr_con[0][0]}\right)\left(\frac{i+j+k+l}{i+j}\right) \frac{d B^{i+j}_j(a)}{da} B^{i+j+k-1}_k(b) B^{i+j+k+l-1}_l(c) \\
& + \left(\sage{da_dr_con[1][0]}\right)\left(\frac{i+j+k+l}{i+j+k}\right) B^{i+j}_j(a) \frac{d B^{i+j+k}_k(b)}{db} B^{i+j+k+l-1}_l(c) \\
& + \left(\sage{da_dr_con[2][0]}\right) B^{i+j}_j(a) B^{i+j+k}_k(b) \frac{d B^{i+j+k+l}_l(c)}{dc},
\end{align*}

\begin{align*}
\frac{\partial B^p_{i,j,k,l}(a,b,c)}{\partial s}
& = \left(\sage{da_dr_con[0][1]}\right)\left(\frac{i+j+k+l}{i+j}\right) \frac{d B^{i+j}_j(a)}{da} B^{i+j+k-1}_k(b) B^{i+j+k+l-1}_l(c) \\
& + \left(\sage{da_dr_con[1][1]}\right)\left(\frac{i+j+k+l}{i+j+k}\right) B^{i+j}_j(a) \frac{d B^{i+j+k}_k(b)}{db} B^{i+j+k+l-1}_l(c) \\
& + \left(\sage{da_dr_con[2][1]}\right) B^{i+j}_j(a) B^{i+j+k}_k(b) \frac{d B^{i+j+k+l}_l(c)}{dc},
\end{align*}

\begin{align*}
\frac{\partial B^p_{i,j,k,l}(a,b,c)}{\partial t}
& = \left(\sage{da_dr_con[0][2]}\right)\left(\frac{i+j+k+l}{i+j}\right) \frac{d B^{i+j}_j(a)}{da} B^{i+j+k-1}_k(b) B^{i+j+k+l-1}_l(c) \\
& + \left(\sage{da_dr_con[1][2]}\right)\left(\frac{i+j+k+l}{i+j+k}\right) B^{i+j}_j(a) \frac{d B^{i+j+k}_k(b)}{db} B^{i+j+k+l-1}_l(c) \\
& + \left(\sage{da_dr_con[2][2]}\right) B^{i+j}_j(a) B^{i+j+k}_k(b) \frac{d B^{i+j+k+l}_l(c)}{dc}.
\end{align*}


%%%%%%%%%%%%%%%%%%%%%%%%%%%%%%%%%%%%%%%%%%%%%%%%%%%%%%%%%%%%%%%%%%%%%%%%%%%%%%%%%%%%%%%%%%%%%%%%%%%%
%\section*{References}

\bibliographystyle{elsarticle-num}
\bibliography{../code.bib}

\end{document}
